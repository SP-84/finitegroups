\documentclass[french]{article}
\usepackage[T1]{fontenc} % Font enconding
\usepackage[french]{babel} % Input encoding in French
\usepackage{mathtools} % Include math symbols
\usepackage{amsfonts} % Include font for math symbols
\usepackage{amsthm} % Change definition layout
\usepackage{amssymb} % Use aleph and beth symbols
\usepackage{enumitem} % Modern item enumeration
\usepackage{csquotes} % Typeset quoted texts in French correctly
\usepackage[backend=biber]{biblatex} % For bibliography usage
\usepackage{tikz} % Draw pictures and illustrate examples
\addbibresource{references.bib}
\usepackage{hyperref} % Select links in table of contents
\hypersetup{ 
    colorlinks,
    citecolor=blue,
    filecolor=black,
    linkcolor=blue,
    urlcolor=blue, 
}

\theoremstyle{definition}
\newtheorem{definition}[subsubsection]{Définition}

\theoremstyle{plain}
\newtheorem{proposition}[subsubsection]{Proposition}

\newtheorem{theorem}[subsubsection]{Théorème}

\theoremstyle{plain}
\newtheorem{corollary}[subsubsection]{Corollaire}

\theoremstyle{plain}
\newtheorem{lemma}[subsubsection]{Lemme}

\theoremstyle{plain}
\newtheorem{remark}[subsubsection]{Remarque}

\theoremstyle{plain}
\newtheorem*{notation}{Notation}

\title{AMS ECUE Algèbre générale : Autour de la théorie des groupes finis}
\author{Auteur :\\
	Guillaume SALLOUM \\ 
	L3 Mathématiques \\ 
	Avignon Université
	\and
	Enseignant : \\
	Philippe BOLLE \\
	PR \\
	Avignon Université}
\date{}

\begin{document}

\maketitle
\begin{abstract}
	Le but de cette AMS est d'introduire certains concepts de théorie des groupes permettant de construire des théorèmes de classification.
	Nous étudierons principalement les théorèmes de Sylow et leurs applications, ainsi que quelques éléments de théorie de la représentation  (lemme de Schur) utiles à la classification.
	Le document est divisé en trois parties dont l'une est en anglais afin de lier les deux AMS.  
\end{abstract}


\tableofcontents
\clearpage
\section{Actions de groupes et théorèmes de Sylow}
\subsection{Actions de groupes}

\par Une idée récurrente en mathématiques est d'étudier comment un objet \textit{agit} sur un autre afin d'obtenir plus d'informations sur les deux. C'est précisement le cas lorsque l'on considère un groupe G agissant sur un ensemble A.

\begin{definition}[Action de groupe]
	Soit G un groupe et A un ensemble quelconque. Une action \textit{à gauche} de G sur A est une application \( f : G \times A \rightarrow A\) qui satisfait :
	\begin{enumerate}[label = (\roman*)]
		\item \(g1.(g2.a) = (g1g2).a \) pour tout \(g1,g2 \in G\), \(a \in A\),
		\item \(1.a = a\) pour tout \(a \in A\)
	\end{enumerate}

	On peut définir de manière équivalente \cite[p. ~43]{dummit2003abstract} une action comme un morphisme \( \varphi : G \rightarrow S_{A} \) de G dans le groupe des permutations de A satisfaisant :
	\begin{equation}
		g.a = \varphi(g)(a) \quad \forall g \in G, \forall a \in A
	\end{equation}
\end{definition}

Un exemple important d'action de G sur lui-même est la \textit{conjugaison}.

\cite{serre1979ens}
\cite{chenevier2024ens}

\begin{align*}
	f : G \times G &\to G \\
	g . a &\mapsto gag^{-1}
\end{align*}
\subsection{Théorèmes de Sylow}

Nous enonçons et prouvons une réciproque partielle au théorème de Lagrange utile pour déterminer si un groupe est simple.

\begin{definition}[p-groupe]
	Soit \( G \) un groupe et \( p \) un nombre premier.
	\begin{enumerate}[label = (\roman*)]
	\item Si \( |G| = p^{\alpha} \) pour un \( \alpha > 0 \), G est un \textit{p-groupe}. Un sous groupe H de G est appelé \textit{p-sous groupe}.
	\item Si \( |G| = p^{\alpha}m \) où m n'est pas un multiple de p, alors un sous groupe d'ordre \( p^{\alpha} \) est appelé \textit{p-sous groupe de Sylow de G}.
	\item L'ensemble des p-sous groupes de Sylow de G est noté \( Syl_{p}(G) \), et le nombre de p-sous groupes de Sylow de G est noté \( n_{p}(G) \).
	\end{enumerate}
\end{definition}

\cite[p.~213]{chen2024napkin}

\section{Vers la classification des groupes finis simples}

\cite{dummit2003abstract}

\section{Elements of representation theory of finite groups}
\subsection{Character theory}

\clearpage
\printbibliography[heading=bibintoc]

\end{document}
