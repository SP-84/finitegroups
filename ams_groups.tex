\documentclass[french]{article}
\usepackage[T1]{fontenc} % Font enconding
\usepackage[french]{babel} % Input encoding in French
\usepackage{mathtools} % Include math symbols
\usepackage{amsfonts} % Include font for math symbols
\usepackage{amsthm} % Change definition layout
%\usepackage{amssymb} % Use aleph and beth symbols
\usepackage{enumitem} % Modern item enumeration
\usepackage{csquotes} % Typeset quoted texts in French correctly
\usepackage[backend=biber,style=alphabetic]{biblatex}
%\usepackage{tikz} % Draw pictures and illustrate examples
\addbibresource{ams_groups.bib}
\usepackage{hyperref} % Select links in table of contents
\hypersetup{ 
    colorlinks,
    citecolor=blue,
    filecolor=black,
    linkcolor=blue,
    urlcolor=blue, 
}
\usepackage[toc,nomain,symbols]{glossaries} % Make a list of symbol used in the text 
\makeglossaries
\newglossaryentry{G}{
	name={\ensuremath{G}},
	sort=G,
	description={Groupe quelconque},
	type=symbols
}
\newglossaryentry{n}{
	name={\ensuremath{H \trianglelefteq G}},
	sort=n,
	description={H est un sous-groupe distingué de G},
	type=symbols
}
\newglossaryentry{sylpg}{
	name={\ensuremath{Syl_{p}({G})}},
	sort=sylpg,
	description={Ensembles des p-sous-groupes de Sylow de G},
	type=symbols
}
\newglossaryentry{leftconj}{
	name={\ensuremath{x^{g}}},
	sort=leftconj,
	description={Conjugué à gauche de x par g},
	type=symbols
}
\newglossaryentry{normalizer}{
	name={\ensuremath{N_{G}(H)}},
	sort=normalizer,
	description={Normalisateur de G dans H},
	type=symbols
}
\newglossaryentry{psubgroupsnb}{
	name={\ensuremath{\alpha_{p}}},
	sort=psubgroupsnb,
	description={Nombre de p-sous-groupes de Sylow de G},
	type=symbols
}
\newglossaryentry{isomorphism}{
	name={\ensuremath{A \cong B}, \ensuremath{A \overset{\varphi}{\cong} B}},
	sort=isomorphism,
	description={A est isomorphe à B, \ensuremath{\varphi} est un isomorphisme de A vers B},
	type=symbols
}

\theoremstyle{definition}
\newtheorem{definition}[subsubsection]{Définition}

\theoremstyle{plain}
\newtheorem{proposition}[subsubsection]{Proposition}

\newtheorem{theorem}[subsubsection]{Théorème}

\theoremstyle{plain}
\newtheorem{corollary}[subsubsection]{Corollaire}

\theoremstyle{plain}
\newtheorem{lemma}[subsubsection]{Lemme}

\theoremstyle{plain}
\newtheorem{remark}[subsubsection]{Remarque}

\theoremstyle{plain}
\newtheorem*{notation}{Notation}

\title{AMS Autour des p-sous-groupes de Sylow}
\author{Samy AMARA, Gabriel PITINO, et Guillaume SALLOUM} 
\date{}

\begin{document}
\maketitle
\begin{abstract}
	Le but de cette AMS est d'étudier les théorèmes de Sylow, qui forment une réciproque partielle au thoérème de Lagrange.
	Nous introduirons quelques notions utiles pour enonçer les théorèmes, puis nous en donnerons deux preuves : la première utilise les actions de groupes, tandis que la deuxième se base sur ????  .
	Enfin nous en donnerons quelques applications, notemment pour la classification des groupes simples finis.% et en théorie de la représentation ?
\end{abstract}


%\setcounter{tocdepth}{2}

\tableofcontents
\clearpage

Dans tout ce qui suit, nous utiliserons les notations suivantes :
\glsaddall
\printglossary[title=Notations,type=symbols,style=long,nonumberlist]

\clearpage


\section{Théorèmes de Sylow}
\subsection{Notions préliminaires}

%\par Une idée récurrente en mathématiques est d'étudier comment un objet \textit{agit} sur un autre afin d'obtenir plus d'informations sur les deux. C'est précisement le cas lorsque l'on considère un groupe \( G \) agissant sur un ensemble \( A \).

\begin{definition}[Action de groupe]
	Soit \( G \) un groupe et \( A \) un ensemble quelconque. Une action \textit{à gauche} de \( G \) sur \( A \) est une application \( f : G \times A \rightarrow A\) qui satisfait :
	\begin{enumerate}[label = (\roman*)]
		\item \(g1.(g2.a) = (g1g2).a \) pour tout \(g1,g2 \in G\), \(a \in A\),
		\item \(1.a = a\) pour tout \(a \in A\)
	\end{enumerate}

	On peut définir de manière équivalente d'après le chapitre 1 du cours une action comme un morphisme \( \varphi : G \rightarrow S_{A} \) de G dans le groupe des permutations de A satisfaisant :
	\begin{equation}
		g.a = \varphi(g)(a) \quad \forall g \in G, \forall a \in A
	\end{equation}
\end{definition}

Un exemple important d'action de \( G \) sur lui-même que nous allons utiliser par la suite est la \textit{conjugaison}.

%\cite{serre1979ens}
%\cite{chenevier2024ens}

\begin{align*}
	f : G \times G &\to G \\
	g . a &\mapsto x^{g} \coloneq gag^{-1}
\end{align*}

\begin{definition}[Normalisateur de H dans G]
	Le normalisateur de \( H \) dans \( G \) est l'ensemble 
	\( N_{G}({H}) \coloneq \{ g \in G \mid gPg^{-1} = P \} \).
	De manière équivalente, c'est le stabilisateur de \( H \) sous l'action de conjugaison.
\end{definition}

\begin{definition}[Sous-ensembles conjugués]
	Soit \( G \) un groupe, \( A \) et \( B \) deux sous-ensembles de \( G \). 
	\( A \) et \( B \) sont dit 
	\textit{conjugués dans G} s'il existe \( g \in G \) tel que \( B = gAg^{-1} \).
	En d'autres termes, \( A \) et \( B \) sont dans le même orbite pour l'action de conjugaison.
\end{definition}


\subsection{Enoncé}

Nous enonçons en premier lieu quelques définitions issues de l'énoncé du sujet (ou de manièr équivalente de \cite[p. ~123 et 139]{dummit2003abstract}) utiles pour poser le théorème.

\begin{definition}[p-groupe]
	Soit \( G \) un groupe et \( p \) un nombre premier.
	\begin{enumerate}[label = (\roman*)]
	\item Si \( |G| = p^{\alpha} \) pour un \( \alpha > 0 \), \( G \) est un \textit{p-groupe}. Un sous groupe H de G est appelé \textit{p-sous groupe}.
	\item Si \( |G| = p^{\alpha}m \) où m n'est pas un multiple de p, alors un sous groupe d'ordre \( p^{\alpha} \) est appelé \textit{p-sous groupe de Sylow de G}.
	\item L'ensemble des p-sous groupes de Sylow de \( G \) est noté \( Syl_{p}(G) \), et le nombre de p-sous groupes de Sylow de \( G \) est noté \( n_{p}(G) \).
	\end{enumerate}
\end{definition}

Plusieurs formulations sont possibles pour les théorèmes, 
nous avons décidé d'adapter celle de l'énoncé du sujet directement en ajoutant un dernier point issu de \cite[p.~215]{chen2024napkin}.

\begin{theorem}[Théorèmes de Sylow]
	Soit \( G \) un groupe d'ordre \( p^{\alpha}m \) où \( p \) est un nombre premier, \( m \) et \( p \) sont premiers entre eux. Alors on a :
	\begin{enumerate}[label={\upshape(\roman*)}]
		\item (Existence) Au moins un p-sous-groupe de Sylow existe, c'est-à-dire que \( \alpha_{p}(G) \geq 1 \) et \( Syl_{p}(G) \neq \emptyset \)
		\item 
		\item Le nombre de p-sous-groupes de Sylow de G est congru à 1 modulo p, i.e. \( \alpha_{p} \equiv 1 [p] \) ou \( \alpha_{p} = 1 + kp \).

	\end{enumerate}

\end{theorem}

\subsection{Première démonstration}

\subsection{Deuxième démonstration}


\clearpage
\section{Applications}
\subsection{Application à la classification des groupes simples finis}

\cite{dummit2003abstract}

\subsection{Example : classifing all groups of order 60 up to isomorphism}


\clearpage

\printbibliography

\end{document}
